\documentclass[aspectratio=169]{beamer}
\usetheme{metropolis}
\metroset{subsectionpage=progressbar}

\usefonttheme{professionalfonts} % 使用系统/自定字体


% === 字体设置 ===
\usepackage[UTF8,scheme=plain,fontset=none]{ctex}
\setCJKmainfont{Source Han Serif CN}[BoldFont={Source Han Serif CN Bold}]
\setCJKsansfont{Source Han Sans CN}[BoldFont={Source Han Sans CN Bold}]
% \setCJKmonofont{Sarasa Mono CN}


% beamer 已加载 hyperref;加 unicode 以支持中文书签
\hypersetup{unicode}

% define paragraph
\providecommand{\paragraph}[1]{\smallskip\textbf{#1}\par}

% 常用包
\usepackage{longtable,booktabs}
\usepackage{amsmath,amssymb}
\usepackage{graphicx}
% \graphicspath{{.}{./figs/}{./images/}{./images_in_paper/}}
\usepackage{caption}
\usepackage{subcaption}
\usepackage{float}
\usepackage{svg}
\usepackage{booktabs}
\usepackage{array}
\usepackage{threeparttable}

% 算法环境(与 beamer 兼容)
\usepackage{algorithm}
\usepackage[noend]{algpseudocode}  % 提供 algorithmic 环境、\State 等
% 可选:微调 algorithmic 缩进
\algrenewcommand\algorithmicindent{0.8em}

% 数学粗体与梯度符号
\usepackage{bm} % \boldsymbol
\newcommand{\bx}{\mathbf{x}}
\newcommand{\bz}{\mathbf{z}}
\newcommand{\bI}{\mathbf{I}}
\newcommand{\bzero}{\mathbf{0}}
\newcommand{\bepsilon}{\boldsymbol{\epsilon}}
\newcommand{\grad}{\nabla}
% 超链接(beamer 已加载 hyperref,这里只补选项)
% \hypersetup{unicode=true}

% 编号风格
\setbeamertemplate{caption}[numbered]
\setbeamertemplate{caption label separator}{.}

\title{扩散模型在图像重建与生成任务中的应用}
\author{李孟霖}
\date{\today}
%---Document Begins---
\begin{document}
\begin{frame}[plain]
  \titlepage
\end{frame}
\section{DDPM}
\begin{frame}
Denoising Diffusion Probabilistic Models(DDPM)通过逐步添加噪声破坏图像,
再用反向过程逐步去噪重建图像。
该模型结合了变分下界优化和Langevin动力学的去噪得分匹配,实现高质量图像生成。
\end{frame}
\subsection{数学原理}

\begin{frame}{Diffusion Model}
  \textbf{前向过程 (Forward Process)}: \\
  给定原始图像 $x_0 \sim q(x_0)$,前向过程逐步添加高斯噪声,构成一个固定的马尔可夫链:
  \[
    q(x_{1:T} | x_0) = \prod_{t=1}^T q(x_t | x_{t-1}), \quad q(x_t | x_{t-1}) = \mathcal{N}(\sqrt{1 - \beta_t} x_{t-1}, \beta_t I)
  \]
  可直接从 $x_0$ 采样出任意时刻 $x_t$:
  \[
    q(x_t | x_0) = \mathcal{N}(\sqrt{\bar{\alpha}_t} x_0, (1 - \bar{\alpha}_t) I)
  \]
  其中 $\alpha_t = 1 - \beta_t$, $\bar{\alpha}_t = \prod_{s=1}^t \alpha_s$

  \vspace{0.5em}
  \textbf{反向过程 (Reverse Process)}: \\
  从纯噪声 $x_T \sim \mathcal{N}(0, I)$ 开始,反向过程学习一个去噪马尔可夫链:
  \[
    p_\theta(x_{0:T}) = p(x_T) \prod_{t=1}^T p_\theta(x_{t-1} | x_t), \quad p_\theta(x_{t-1} | x_t) = \mathcal{N}(\mu_\theta(x_t, t), \Sigma_\theta(x_t, t))
  \]
  通过神经网络学习均值和方差,实现高质量图像的逐步重建。
\end{frame}

\begin{frame}{DDPM 损失函数构造与简化}

\textbf{目标:} 利用变分下界(ELBO),设计可优化的训练损失函数,以学习从噪声 $x_T$ 逐步生成图像 $x_0$ 的反向过程。

\textbf{1. 变分下界(ELBO)分解:}
\[
\log p_\theta(x_0) \geq \text{ELBO} = L_T + \sum_{t=2}^T L_{t-1} + L_0
\]
- $L_T$ 是固定常数,可忽略
- $L_{t-1}$ 是主要的训练目标项(KL 散度)
- $L_0$ 是重建项(有时也忽略)

\textbf{2. 反向过程建模为高斯分布:}
\[
p_\theta(x_{t-1}|x_t) = \mathcal{N}(\mu_\theta(x_t, t), \sigma_t^2 I)
\]
- 其中 $\sigma_t^2$ 为固定超参数,不学习
- $\mu_\theta$ 为模型(如 U-Net)输出
\end{frame}
\begin{frame}{DDPM 损失函数构造与简化}
\textbf{3. 损失函数重写为预测后验均值:}
\[
L_{t-1} = \mathbb{E} \left[ \frac{1}{2\sigma_t^2} \| \mu_\theta(x_t, t) - \tilde{\mu}_t(x_t, x_0) \|^2 \right] + C
\]

\textbf{4. 使用重参数化将问题转化为噪声预测:}
\[
x_t = \sqrt{\bar{\alpha}_t} x_0 + \sqrt{1 - \bar{\alpha}_t} \epsilon, \quad \epsilon \sim \mathcal{N}(0, I)
\]

\textbf{最终损失函数简化为:}
\[
\mathcal{L}_{\text{simple}} = \mathbb{E}_{x_0, \epsilon, t} \left[ \left\| \epsilon - \epsilon_\theta(x_t, t) \right\|^2 \right]
\]

\textbf{结论:} 模型训练目标是拟合噪声 $\epsilon$,本质上是一个时间条件的\textbf{去噪任务}。

\end{frame}

\begin{frame}{algorithm}

\algrenewcommand\algorithmicindent{0.5em}%
\begin{figure}[t]
\begin{minipage}[t]{0.495\textwidth}
\begin{algorithm}[H]
  \caption{Training} \label{alg:training}
  \small
  \begin{algorithmic}[1]
    \Repeat
      \State $\bx_0 \sim q(\bx_0)$
      \State $t \sim \mathrm{Uniform}(\{1, \dotsc, T\})$
      \State $\bepsilon\sim\mathcal{N}(\bzero,\bI)$
      \State Take gradient descent step on
      \Statex $\qquad \grad_\theta \left\| \bepsilon - \bepsilon_\theta(\sqrt{\bar\alpha_t} \bx_0 + \sqrt{1-\bar\alpha_t}\bepsilon, t) \right\|^2$
    \Until{converged}
  \end{algorithmic}
\end{algorithm}
\end{minipage}
\hfill
\begin{minipage}[t]{0.495\textwidth}
\begin{algorithm}[H]
  \caption{Sampling} \label{alg:sampling}
  \small
  \begin{algorithmic}[1]
    \vspace{.04in}
    \State $\bx_T \sim \mathcal{N}(\bzero, \bI)$
    \For{$t=T, \dotsc, 1$}
      \State $\bz \sim \mathcal{N}(\bzero, \bI)$ if $t > 1$, else $\bz = \bzero$
      \State $\bx_{t-1} = \frac{1}{\sqrt{\alpha_t}}\left( \bx_t - \frac{1-\alpha_t}{\sqrt{1-\bar\alpha_t}} \bepsilon_\theta(\bx_t, t) \right) + \sigma_t \bz$
    \EndFor
    \State \textbf{return} $\bx_0$
    \vspace{.04in}
  \end{algorithmic}
\end{algorithm}
\end{minipage}
\vspace{-1em}
\end{figure}
\end{frame}

\subsection{模型结构}
\begin{frame}{DDPM 模型结构(U-Net)}
  \begin{figure}[H]
    \centering
    \includegraphics[width=0.9\linewidth]{images/DDPM_UNet.png}
    \caption{DDPM 中用于噪声预测的 U-Net 架构示意图}
  \end{figure}
\end{frame}

\begin{frame}{时间步条件的注入方式}
  \begin{figure}[H]
    \centering
    \includegraphics[width=0.8\linewidth]{images/DDPM_down-up-block.png}
    \caption{U-Net 层内部通过 Add 操作将时间步嵌入融合到特征图中}
  \end{figure}
\end{frame}

\end{document}
